%#Split: 02_purpose_plan  
%#PieceName: p02_purpose_plan
\input{pieces/p02_purpose_plan_00}
\section{研究目的・内容等}
%    <<最大 2ページ>>

%s02_purpose_plan_dcpd
%begin 研究目的と研究計画short留意事項なし ====================
%begin 研究計画における研究目的、研究方法、研究内容 ====================
\graysubsection{研究計画における研究目的、研究方法、研究内容}
\subsubsection*{研究目的}
フェイクニュースの早期自動検出を日本語で実現する

\subsubsection*{研究方法・研究内容}
新たに日本語の記事・真偽を含むデータセットを作成する

弱教師あり学習によってラベル不足を補うモデルを構築する

トピックに左右されない汎化性能向上を模索する

%end 研究計画における研究目的、研究方法、研究内容 ====================

%begin どのような計画で、何を、どこまで明らかにしようとするのか ====================
\graysubsection{どのような計画で、何を、どこまで明らかにしようとするのか}
\subsubsection*{日本語の記事・真偽を含むデータセットを作成する}
ファクトチェック済記事とそうでない記事を収集する

\subsubsection*{弱教師あり学習によってラベル不足を補うモデルを構築する}


\subsubsection*{トピックに左右されない汎化性能向上を模索する}
的確な指摘を行うユーザの反応を評価する

%end どのような計画で、何を、どこまで明らかにしようとするのか ====================

%begin 研究の特色・独創的な点 ====================
\graysubsection{研究の特色・独創的な点}
\subsubsection*{本研究の特色}
日本語で自動検出を行う

早期検出を行う

変化する社会情勢に対しロバストである

\subsubsection*{先行研究との比較}
英語に偏重している

日本語データセットが不足している

\subsubsection*{予想されるインパクト・将来の見通し}
SNS利用者への注意喚起に活用可能

ファクトチェック支援システムへの活用

騙されて社会的損害や風評被害が発生するケースを未然に防ぐ

%end 研究の特色・独創的な点 ====================

%begin 申請者が担当する部分 ====================
\graysubsection{申請者が担当する部分}
ぜんぶ
%end 申請者が担当する部分 ====================

%begin 受入研究機関と異なる研究機関での研究従事計画 ====================
\graysubsection{受入研究機関と異なる研究機関での研究従事計画}
たりん
%end 受入研究機関と異なる研究機関での研究従事計画 ====================

\vspace{1cm}
%\begin{thebibliography}{99}
%\end{thebibliography}
%end 研究目的と研究計画short留意事項なし ====================
\input{pieces/p02_purpose_plan_01}

