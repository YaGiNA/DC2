%#Split: 04_abilities  
%#PieceName: p04_abilities
% p04_abilities_00.tex
\KLBeginSubjectWithHeaderCommands{04}{4}{研究遂行力の自己分析}{2}{F}{}{\DCPDFirstSubjectPageStyle}{\DCPDDefaultPageStyle}

\section{研究遂行力の自己分析}
%    <<最大 2ページ>>

% s14_abilities
%begin 自己分析 ====================
%\DCPDInstructionsA\\% <-- 留意事項:これは消すか、コメントアウトしてください。
\noindent
\graybf{(1) 研究に関する自身の強み}
\subsubsection*{自ら抱いた問題意識を出発点に研究を行う主体性}

\subsubsection*{貪欲な海外論文調査に裏打ちされた状況把握能力}

\subsubsection*{産学問わない活動で培ったプログラミング能力で実現される実装能力}

\subsubsection*{プラットフォームを問わず議論を活発に行えるコミュニケーション能力}

\subsubsection*{相手が小学生でも物事を分かりやすく伝えることができるプレゼン能力}
%\DCPDInstructionsB% <-- 留意事項:これは消すか、コメントアウトしてください。

\vspace{5mm}
\noindent
\graybf{(2) 今後研究者として更なる発展のため必要と考えている要素}
\vspace{-5mm}
\subsubsection*{要素Ⅰ: 学術的成果と社会問題の最前線の間にあるギャップを埋めるための発想と問題解決力}
フェイクニュースの自動検出を行う研究は世界的には広く行われており、
それぞれが独自の発想を追加している。
この独自の発想の付加には、1つの技術分野に絞らず他分野から得た知見がもたらす。
そのため技術面では自然言語処理に限らず、利用者の拡散を考慮するためのグラフネットワークや、
既知の情報を利用するためのナレッジグラフ技術など、
\underline{幅広い分野の研究に論文を通して触れる}必要がある。

\vspace{-5mm}
\subsubsection*{要素Ⅱ: 多彩な分野や言語・地域圏の研究者らと活発な議論を交わす能力}
要素Ⅰの実現には、論文のみならず実際に議論を交わすことで更に深い理解を得ることが重要である。
また海外で研究が活発に行われていることから、知見のアップデートも頻繁に行うことも必要である。
よって\underline{分野・言語問わず多くの研究者達とプラットフォームを問わない深い議論が研究の発展をもたらす}と考える。

\vspace{-5mm}
\subsubsection*{要素Ⅲ: 研究で得られた成果をどんな聞き手でも分かりやすく伝えられる表現力}
新型コロナウイルス感染症蔓延の影響もあり、発表の機会や形式が大きく制限されたまま修士研究を終えた。
オンライン形式での発表での経験を積めた一方、人前で発表する機会は未だにない。
以上から\underline{場所を問わず誰が相手でも研究を分かりやすく伝える経験を積む}必要が例年以上に必要と考える。
%end 自己分析 ====================

% p04_abilities_01.tex
\KLEndSubject{F}


