%#Split: 04_abilities  
%#PieceName: p04_abilities
% p04_abilities_00.tex
\KLBeginSubjectWithHeaderCommands{04}{4}{研究遂行力の自己分析}{2}{F}{}{\DCPDFirstSubjectPageStyle}{\DCPDDefaultPageStyle}

\section{研究遂行力の自己分析}
%    <<最大 2ページ>>

% s14_abilities
%begin 自己分析 ====================
%\DCPDInstructionsA\\% <-- 留意事項:これは消すか、コメントアウトしてください。
\noindent
\graybf{(1) 研究に関する自身の強み}

\sbsbsection{自分の問題意識から研究を発展させる主体性}
私は高校3年次にバイオインフォマティクス(生命情報学)分野の研究活動として初めてプログラミングを行いWebクローラーを開発した(成果\ref{enum:hg})。
その後大学でコンピュータサイエンスの経験を積む中で、
%熊本地震や米国大統領選挙でフェイクニュース問題が多く取り沙汰されるようになった。
%誤った情報が広まって風評被害が出る事例は古今東西起きているものの、
%ことSNSが普及した現代社会では共有によって拡散のスピードが速く広くなる点に危機感を抱いた。
社会情勢の変化から自動で誤った情報を検出できないか考えるようになった。
学部〜修士課程でこの問題意識からフェイクニュース自動検出をテーマに据え、
指導教員を始め多くの研究者から助言を受けつつ\underline{主体的に研究を進めてきた}。
英文記事が対象のため、
\underline{積極的に海外学会を中心に成果を発表し}、\underline{研究内容や今後の発展について議論}を交わした。
\underline{既に1回査読付き海外IEEE学会で口頭発表}し(成果\ref{enum:ines})、
\underline{国内研究会でも3回口頭発表}を行った(成果\ref{enum:MACC}〜\ref{enum:NLC})。

\sbsbsection{豊富な海外論文調査による状況把握能力}
私が学部〜修士で行ってきたテーマ(成果\ref{enum:ines}〜\ref{enum:NLC})は、
特に海外で盛んに行われている研究である。
このため積極的に英語論文の調査を行い、
その研究内容を研究室内部のみならずときに他大学の学生へ共有することで、
フェイクニュースの自動検出を行う研究における現在の状況を仔細に把握できるようになった。
実際に学部論文(内容は成果\ref{enum:MACC}に近い)で引用した文献26件のうち24件が英語論文や記事であり、
修士論文(内容は成果\ref{enum:NLC}に近い)で引用した文献76件のうち74件が英語論文や記事であった。
このように私は\underline{貪欲な海外論文調査に裏打ちされた状況把握能力}があると考える。

\sbsbsection{産学問わない活動から培われた実装能力}
私は\underline{産学問わず}プログラミング活動を行ってきた。
最初のプログラミング開発(成果\ref{enum:hg})では、実装に適した言語の選定から、独習〜実装、
そしてポスター作成まで一貫して大学合格発表直後から発表会までの2週間で独力で行った。
このプログラミング能力は大学入学直後も発揮し続け、
産業界ではこれまで2社で3ヶ月以上継続したエンジニア活動を行っている(成果\ref{enum:amelieff}, \ref{enum:jict})ほか、
1〜2ヶ月に及ぶ短期エンジニアインターンシップも2社で行った(成果\ref{enum:fixstars}, \ref{enum:ak})。
一方学部研究(成果\ref{enum:MACC})では3カテゴリ分類モデルの実装を独力で行った。
また修士研究(成果\ref{enum:ines}, \ref{enum:SMASH}, \ref{enum:NLC})では、
既存のフェイクニュース生成モデルGrover\cite{grover}のソースコードを改変する形で記事ではなくコメントを生成するシステムを構築した。
よって私には豊富な開発経験がもたらす実装能力があると確信する。

\sbsbsection{プレゼン・コミュニケーション能力}
\underline{研究会に限らずリモート開催された海外学会にも積極的に参加}し、
研究者達とのコミュニケーションを積極的に行った。
また他大学との合同ゼミにも参加し、別分野の研究に対する理解を深めている。

私は大学入りしてから\underline{小中高校生を対象としたプログラミング教室の立ち上げから関与を続け}、
講師としても2年半かけて活動を続けた(成果\ref{enum:prog})。
具体的には教える言語(Python)の習得を目的とした輪講に参加し、
講師として開講から2年間にわたり毎週受講生の学習のメンタリングを行った。
受講生が分からないと申告した部分は実際にその部分が不明なのか、それとも前提とする部分から既に不明なのか、
丁寧な聞き込みから特定を行い指導することで自学自習が進みやすくなるよう意識し、
また受講生が自主制作したいプログラムが現在の技量に対して高度過ぎる場合は、
段階となりそうな例を示して開発するよう指導した。
こういった経験で、\underline{他者の視点に立った説明が必要だと強く認識}した。

以上から\underline{相手が小学生でも物事を分かりやすく伝えられるプレゼン能力}があると考える。
%\DCPDInstructionsB% <-- 留意事項:これは消すか、コメントアウトしてください。

\sbsbsection{成果: 国際会議における発表}
(以下1件 査読あり・論文及び口頭発表)
\vspace{-4mm}
\begin{enumerate}
    \setlength{\parskip}{0cm}
    \setlength{\itemsep}{0cm}
    \item \underline{$\circ$ Yuta Yanagi}, Ryouhei Orihara, Yuichi Sei, Yasuyuki Tahara, and Akihiko Ohsuga.\\
    ``Fake news detection with generated comments for news articles''.\\
    The 24th IEEE International Conference on Intelligent Engineering Systems 2020,\\
    (Reykjavík, Iceland) Virtual event due to COVID-19, July 2020. \label{enum:ines}
\end{enumerate}

\sbsbsection{成果: 国内学会やシンポジウムにおける発表}
(以下3件 査読なし・口頭発表)
\vspace{-3mm}
\begin{enumerate}
    \setcounter{enumi}{1}
    \setlength{\parskip}{0cm}
    \setlength{\itemsep}{0cm}
    \item \underline{$\circ$ 栁裕太}、田原康之、清雄一、大須賀昭彦\\
        「画像付きフェイクニュースとジョークニュースの検出・分類に向けた機械学習モデルの検討」、\\
        日本ソフトウェア科学会 2018年度 MACC研究発表会、
        大分、2019年3月 \label{enum:MACC}
    \item \underline{$\circ$ 栁裕太}、折原良平、清雄一、田原康之、大須賀昭彦\\
        「記事へのコメント生成によるフェイクニュースの早期検出」\\
        Symposium on Multi Agent Systems for Harmonization 2020 (SMASH20), Online, 2020年8月 \label{enum:SMASH}
    \item \underline{$\circ$ 栁裕太}、折原良平、清雄一、田原康之、大須賀昭彦\\
        「フェイクニュースの早期自動検出に向けたニュース記事コメント生成モデルの提案」、\\
        言語理解とコミュニケーション研究会(NLC) 第17回テキストアナリティクス・シンポジウム、\\
        Online, 2021年2月 \label{enum:NLC}
\end{enumerate}
(以下1件 査読なし・ポスター発表)
\vspace{-3mm}
\begin{enumerate}
    \setcounter{enumi}{4}
    \setlength{\parskip}{0cm}
    \setlength{\itemsep}{0cm}
    \item \underline{$\circ$ 栁裕太}、葛西透麿、 森谷薫平、神谷岳洋、藤原徹、木村健太、榎本裕介\\
        「CaD428の変異遺伝子の機能解析ツールの汎用化」、\\
        広尾学園高校医進・サイエンスコース研究成果報告会、
        東京、2015年3月 \label{enum:hg}
\end{enumerate}

\sbsbsection{成果: 学外活動歴}
\begin{enumerate}
    \setcounter{enumi}{5}
    \setlength{\parskip}{0cm}
    \setlength{\itemsep}{0cm}
    \item UECプログラミング教室講師、2016年5月〜2018年3月\cite{uecprog} \label{enum:prog}
    \item アメリエフ株式会社におけるエンジニア活動、2016年8月〜2018年3月\cite{amelieff} \label{enum:amelieff}
    \item 株式会社justInCase Technologiesにおけるエンジニア活動、2018年10月〜活動中\cite{jic-tech} \label{enum:jict}
    \item 株式会社フィックスターズにおける短期エンジニアインターンシップ活動、2019年7〜8月 \label{enum:fixstars}
    \item 株式会社アカツキにおける短期バックエンドエンジニアインターンシップ活動、2020年7月 \label{enum:ak}
\end{enumerate}

\vspace{5mm}
\noindent
\graybf{(2) 今後研究者として更なる発展のため必要と考えている要素}
\sbsbsection{要素Ⅰ: 学術的成果と社会問題の最前線の間にあるギャップを埋めるための発想力}
フェイクニュースの自動検出を行う研究は世界的に広く行われており、
それぞれが独自の発想を追加している。
この付加する独自の発想は、ときに他分野から得た知見がもたらす。
よって技術面では自然言語処理に限らず、利用者の拡散を考慮するためのグラフネットワークや、
既知の情報を利用するためのナレッジグラフ技術など、
\underline{幅広い分野の研究に論文を通して触れる}必要がある。
具体的にはフェイクニュースの自動検出に限らず、自然言語処理や画像処理、
そしてナレッジグラフなど、関連のありそうな分野全体に広げることで実現できると考えている。

\sbsbsection{要素Ⅱ: 多彩な分野や言語・地域圏の研究者らと活発な議論を交わす能力}
要素Ⅰの実現には、論文のみならず実際に議論を交わすことで更に深い理解を得ることが重要である。
また海外で研究が活発に行われていることから、知見のアップデートも頻繁に行うことも必要である。
よって\underline{分野・言語問わず多くの研究者達と場を問わない深い議論が研究の発展をもたらす}と考える。

\sbsbsection{要素Ⅲ: 研究で得られた成果をどんな聞き手でも分かりやすく伝えられる表現力}
新型コロナウイルス感染症蔓延の影響もあり、発表の機会や形式が大きく制限されたまま修士研究を終えた。
オンライン形式での発表での経験を積めた一方、人前で発表する機会はあまり多くの経験を積められないままである。
それゆえ、\underline{誰が相手でも研究を分かりやすく伝える経験を積む}必要が例年以上に必要と考える。
%end 自己分析 ====================

%\vspace{1cm}
{\footnotesize
	\begin{thebibliography}{99}
		\setcounter{enumiv}{13}
		\bibitem{grover} R. Zellers, \textit{et al. NeurIPS 2019}
        \bibitem{uecprog} 安部博文, 【第1期子供のためのプログラミング教室(4)記録】, 国立大学法人電気通信大学インキュベーション施設, 2016年5月29日(最終閲覧日 2021年5月3日) \url{http://www.uecincu.com/programming/programming_160529.html}
        \bibitem{amelieff} アメリエフ株式会社「4月21日(金)「医療ビッグデータを活用して世界を変える! 学生インターンMeetup 2017春」開催のお知らせ」, 2017年4月7日(最終閲覧日 2021年5月3日) \url{https://amelieff.jp/170407/}
        \bibitem{jic-tech} 「株式会社 justInCaseTechnologies | 保険を変える保険テック会社」, 2020年12月1日(最終閲覧日 2021年5月3日) \url{https://justincase-tech.com/}
	\end{thebibliography}
}
% p04_abilities_01.tex
\KLEndSubject{F}


