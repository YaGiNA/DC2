%#Split: 04_abilities  
%#PieceName: p04_abilities
\input{pieces/p04_abilities_00}
\section{研究遂行力の自己分析}
%    <<最大 2ページ>>

% s14_abilities
%begin 自己分析 ====================
%\DCPDInstructionsA\\% <-- 留意事項:これは消すか、コメントアウトしてください。
\noindent
\graybf{(1) 研究に関する自身の強み}
\subsubsection*{自ら抱いた問題意識を出発点に研究を行う主体性}

\subsubsection*{貪欲な海外論文調査に裏打ちされた状況把握能力}

\subsubsection*{産学問わない活動で培ったプログラミング能力で実現される実装能力}

\subsubsection*{プラットフォームを問わず議論を活発に行えるコミュニケーション能力}

\subsubsection*{相手が小学生でも物事を分かりやすく伝えることができるプレゼン能力}
%\DCPDInstructionsB% <-- 留意事項:これは消すか、コメントアウトしてください。

\vspace{5mm}
\noindent
\graybf{(2) 今後研究者として更なる発展のため必要と考えている要素}
\subsubsection*{要素1: 学術的成果と社会問題の最前線の間にあるギャップを埋めるための発想と問題解決力}

\subsubsection*{要素2: 多彩な分野や言語・地域圏の研究者らと活発な議論を交わす能力}

\subsubsection*{要素3: 研究で得られた成果をどんな聞き手でも分かりやすく伝えられる表現力}
%end 自己分析 ====================

\input{pieces/p04_abilities_01}

