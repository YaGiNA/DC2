%#Split: 01_background  
%#PieceName: p01_background
% p01_background_00.tex
\KLBeginSubjectWithHeaderCommands{01}{2}{研究の位置づけ}{1}{F}{3}{\DCPDVeryFirstPageStyle}{\DCPDDefaultPageStyle}

\section{研究の位置づけ}
%    <<最大 1ページ>>

%s03_background
%begin 背景: 当該分野の状況 ====================
\graysubsection{当該分野の状況: フェイクニュースの自動検出}
SNSの発展で情報を迅速かつ大量に取得・共有が容易になった一方、
悪意により他人を騙すために作られた\textbf{フェイクニュース}も拡散されやすくなった。
特に2020年からCOVID-19の影響による誤情報の拡散であるインフォデミックにより、
国内外でメタノール飲用による死亡事故\cite{iraninfo}といった事象が報告された。
以上から騙された人々により社会的損害が起きるため、
\bfundl{フェイクニュース拡散の早期抑制が必要である}\cite{snsinfo}。
%end 背景: 当該分野の状況 ====================

%begin 背景: 課題 ====================
\graysubsection{課題}
%取り上げるのは主にこの3つ
%\begin{itemize}
%    \item 英語偏重(日本語の研究が少ない)
%    \item 早期検出
%    \item 汎化性能不足
%\end{itemize}
フェイクニュースの検出作業には、有識者が調査する\textbf{ファクトチェック}がある。
これは属人的な作業で、拡散されてから着手されるため、結果公表まで\bfundl{時間がかかり拡散抑制にはならない}。
そのため、自動でフェイクニュースを検出するべく、深層学習によって
ニュースの内容やユーザの反応、そして真偽等を入力に学習する手法がある。
しかしそれらの研究は正解ラベルとして多量のファクトチェック結果を必要とし、
更にファクトチェックが活発な地域差の影響でデータセットが英語に集中している。
もし日本語を対象とした場合、ファクトチェック結果が不足しているためラベル付きデータセットによる\bfundl{教師あり学習ができない}。
%end 背景: 課題 ====================

%begin 着想に至った経緯 ====================
\graysubsection{本研究計画の着想に至った経緯}
卒論から一貫してフェイクニュース自動検出の研究をやってきた

発表を通して日本語を対象とした研究への期待感が大きい

日本語は英語に比べファクトチェックが盛んではない

ラベル不足を補うために弱教師あり学習に着目した
(弱教師あり学習の説明が必要)
%end 着想に至った経緯 ====================

\vspace{1cm}
\begin{thebibliography}{99}
    \bibitem{iraninfo} Hassanian-Moghaddam, Hossein, et al. ``Double trouble: methanol outbreak in the wake of the COVID-19 pandemic in Iran—a cross-sectional assessment.'' \textit{Critical Care} 24.1 (2020): 1-3.
    \bibitem{snsinfo} Tasnim, Samia, Md Mahbub Hossain, and Hoimonty Mazumder. ``Impact of rumors and misinformation on COVID-19 in social media.'' \textit{Journal of preventive medicine and public health} 53.3 (2020): 171-174.
\end{thebibliography}
% p01_background_01.tex
\KLEndSubject{F}


