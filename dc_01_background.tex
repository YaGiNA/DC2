%#Split: 01_background  
%#PieceName: p01_background
\input{pieces/p01_background_00}
\section{研究の位置づけ}
%    <<最大 1ページ>>

%s03_background
%begin 背景: 当該分野の状況 ====================
\graysubsection{当該分野の状況: フェイクニュースの自動検出}
冒頭に研究モデルの概要を書く

背景の部分はDC1の申請書をベースに

新たにコロナにも触れる
%end 背景: 当該分野の状況 ====================

%begin 背景: 課題 ====================
\graysubsection{課題}
取り上げるのは主にこの3つ
\begin{itemize}
    \item 英語偏重(日本語の研究が少ない)
    \item 早期検出
    \item 汎化性能不足
\end{itemize}

%end 背景: 課題 ====================

%begin 着想に至った経緯 ====================
\graysubsection{本研究計画の着想に至った経緯}
卒論から一貫してフェイクニュース自動検出の研究をやってきた

発表を通して日本語を対象とした研究への期待感が大きい

日本語は英語に比べファクトチェックが盛んではない

ラベル不足を補うために弱教師あり学習に着目した
(弱教師あり学習の説明が必要)
%end 着想に至った経緯 ====================
\vspace{1cm}
%\begin{thebibliography}{99}
%\end{thebibliography}
\input{pieces/p01_background_01}

