%#Split: 01_background  
%#PieceName: p01_background
\input{pieces/p01_background_00}
\section{研究の位置づけ}
%    <<最大 1ページ>>

%s03_background
%begin 背景: 当該分野の状況 ====================
\graysubsection{当該分野の状況: フェイクニュースの自動検出}

\setlength\intextsep{0pt}
\setlength\textfloatsep{0pt}
\begin{wrapfigure}{r}[10pt]{0.3\linewidth}
%    \vspace{-5mm}
    \centering
    \includegraphics[width=0.3\textwidth]{figs/base_model.pdf}
    \vspace{-1cm} 
    \caption{フェイクニュース自動検出の基本的な流れ}
    \label{fig:objects}
\end{wrapfigure}
SNSの発展で情報を迅速かつ大量に取得・共有が容易になった一方、
悪意により他人を騙すために作られた\textbf{フェイクニュース}も拡散されやすくなった。
特に2020年からCOVID-19の影響による誤情報の拡散であるインフォデミックにより、
メタノール飲用による死亡事故\cite{iraninfo}といった事象が報告された。
以上から騙された人々により社会的損害が起きるため、
\underline{フェイクニュース拡散の早期抑制が必要である}\cite{snsinfo}。

フェイクニュース検出へ有識者が調査する\textbf{ファクトチェック}がある。
これは拡散ののち着手されるため、\underline{拡散抑制にはならない}。
そのため、自動でフェイクニュースを検出するべく深層学習によって
ファクトチェック結果をラベルとして、記事内容や利用者の反応から教師あり学習で自動検出する研究がある\cite{Wang:2018:EEA:3219819.3219903}。
%end 背景: 当該分野の状況 ====================

%begin 背景: 課題 ====================
\graysubsection{課題}
フェイクニュース自動検出が抱える課題は以下の通りである。

\begin{description}
    %\vspace{-5mm}
    \setlength{\parskip}{0cm}
    \setlength{\itemsep}{0cm}
    \item[ニュースのタイムリー性] %\mbox{}\\
        ニュースという性質上、時間経過で扱われる情報が徐々に古くなりフェイクニュースの傾向変化への対応が難しくなる。
        先行研究によると学習・検証で入力するニュースの出来事を変えると検出性能が劣化する\cite{Wang:2018:EEA:3219819.3219903}。
        よって\underline{継続してデータセットを拡大する仕組みが必要}である。
    \item[SNSプラットフォームへの依存性] %\mbox{}\\
        SNS上で利用者からの反応を取得する場合、その形式は取得元のSNSプラットフォームに依存する。
        今後主流となるSNSが変わった場合、利用者層や時代の違いにより\underline{既存のデータでは対応できない}。
    \item[早期検出と正確性の両立] %\mbox{}\\
        記事内容に加えて利用者の反応(RT, コメント等)を扱うと検出性能が改善する\cite{Wu:2018:TFF:3159652.3159677}一方、
        利用者の反応を十分に得るには時間がかかり、\underline{高い正確性と早期検出を両立できない}。
    \item[日本語データセット不足] %\mbox{}\\
        深層学習による検出は、正解ラベルとして多量のファクトチェック結果を要する。
        このファクトチェックが活発な地域差の影響でデータセットが\textbf{英語に集中}\cite{fakenewsnet}している。
        もし\textbf{日本語を対象}とした場合、ラベル不足の影響により\underline{教師あり学習ができない}。
    \end{description}
%end 背景: 課題 ====================

%begin 着想に至った経緯 ====================
\graysubsection{本研究計画の着想に至った経緯}
私は修士課程で\underline{英文フェイクニュース早期検出の研究を行った}。
記事に対する利用者のコメントが検出に有用とする先行研究\cite{defend}をベースに、
早期検出を想定して\underline{少ないコメントから更にコメント内容を自動生成}して検出するモデルを実装した。
実験にてコメントを生成した上で分類することでより多くのフェイクニュース検出を実現した(査読付き海外IEEE学会 発表済\cite{ines})。

一方、研究が進むつれて社会変化の激しさを実感した。
ニュースのトピックは日を追うごとに変化すると同時にフェイクニュースの内容も変わる上、
ユーザの反応が現れるSNSプラットフォームも近年は新しいサービスが提供されている。
この影響でこれまでの記事+SNS上の反応を扱う検出形式では、
データセット作成・モデル提供のみでは\underline{時代の変化に対応できない}。
よって\underline{記事を継続して収集する枠組み作り}に併せ、
ユーザの反応を\underline{SNSプラットフォームに左右されない形}で得る点の重要性に着目した。
%一方、国内研究会で発表したところ想定以上に日本語での実現に対する期待を受けた。
%日本は英語圏に比べファクトチェックされた記事が少なく、
%\underline{データセットを作ってモデルを実装するにはラベルが足りない}。
%このラベル不足を補う方法として、少量のラベル付き記事と多量のラベルなし記事に利用者の初期反応から
%弱いアノテーションを付加して学習を行う弱教師あり学習を行う研究\cite{mwss}に着目した。
%日本語で同じ構成のデータセットを作成し、分類を行うモデルを実装することで実現可能と考えた。

%end 着想に至った経緯 ====================

{\footnotesize 
%\vspace{1cm}
\begin{twobibliography}{99}
    \setlength{\parskip}{0cm}
    \setlength{\itemsep}{0cm}
    \bibitem{iraninfo} H. H-M, \textit{et al.} \textit{Critical Care} 24.1 2020: 1-3.
    \bibitem{snsinfo} S. Tasnim, \textit{et al.} \textit{JPMPH} 53.3 2020: 171-174.
    \bibitem{Wang:2018:EEA:3219819.3219903} Y. Wang, \textit{et al.} \textit{KDD'18}, pp. 849-857. 2018.
    \bibitem{Wu:2018:TFF:3159652.3159677} L. Wu \& H. Liu. \textit{WSDM '18}  pp. 637-645, 2018.
    \bibitem{fakenewsnet} K. Shu,\textit{et al.} \textit{Big Data 8.3} 2020: 171-188.
    \bibitem{coviddiff} Y. Bang, \textit{et al.} \textit{arXiv preprint arXiv:2101.03841} 2021.
    \bibitem{defend} K. Shu, \textit{et al.} \textit{KDD'19} 395–405, 2019.
    \bibitem{ines} Y. Yuta, \textit{et al.} \textit{INES}. 2020.
    \bibitem{mwss} K. Shu, \textit{et al.}  \textit{arXiv preprint arXiv:2004.01732} 2020.
\end{twobibliography}
}
\input{pieces/p01_background_01}

