%#Split: 05_my_ambitions  
%#PieceName: p05_my_ambitions
% p05_my_ambitions_00.tex
\KLBeginSubjectWithHeaderCommands{05}{5}{目指す研究者像等}{1}{F}{}{\DCPDFirstSubjectPageStyle}{\DCPDDefaultPageStyle}

\section{目指す研究者像等}
%    <<最大 1ページ>>

% s17_my_ambitions
\noindent
\graybf{(1)目指す研究者像 {\footnotesize ※目指す研究者像に向けて身に付けるべき資質も含め記入してください。}}

%begin 目指す研究者像 ====================
自分の興味のある分野を研ぎ究めると同時に社会問題を解決して人々の生活を幸せにしたい

嘘の情報に騙されて誤った風評が残り不幸になる人を0にしたい

ファクトチェックでは嘘は嘘であると騙された人を相手に分かりやすく説明することが重要である

自己完結のみならず成果を他人に伝えるまでが研究である

%end 目指す研究者像 ====================

\vspace{5mm}
\noindent
\graybf{(2)上記の「目指す研究者像」に向けて、特別研究員の採用期間中に行う研究活動の位置づけ}

%begin 研究活動の位置づけ ====================
特別研究員の採用期間中に行う研究活動のなか、
4-(2)で挙げた今後研究者としてさらなる発展のため必要と考えている要素の習得を通して、
\underline{学術研究で得た知見を直接SNS利用者を含む日本社会に還元}する研究者を目指す。
その実現に向け、査読付き国際会議ないしは国際論文誌への論文発表をはじめ、
国内・国際会議での口頭発表も積極的に行う。
また、自然言語処理コミュニティに限らず国内ニュースメディアと積極的に連携を行い、
フェイクニュースの自動検出に関連した共同研究の実現が理想である。

特別研究員として研鑽を重ねていき、
現状の研究への新たな発想を追加し、実現に向けて幅広い人々と議論を重ね、得られた成果を端的に説明することが、
\underline{能動的に一貫して社会課題を解決へ自ら導く研究者}として大成する。
その実現の大きな足がかりが本研究計画である。
%end 研究活動の位置づけ ====================

% p05_my_ambitions_01.tex
\KLEndSubject{F}


