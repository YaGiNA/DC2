%#Split: 05_my_ambitions  
%#PieceName: p05_my_ambitions
\input{pieces/p05_my_ambitions_00}
\section{目指す研究者像等}
%    <<最大 1ページ>>
\begin{spacing}{1.08}
% s17_my_ambitions
\noindent
\graybf{(1)目指す研究者像} {\footnotesize ※目指す研究者像に向けて身に付けるべき資質も含め記入してください。}

%begin 目指す研究者像 ====================
私は幼少期からコンピュータに触れる機会が多かった影響で、コンピュータサイエンスやプログラミングに興味を持った。
その後広尾学園高校で初めてプログラミングを含む研究活動を通し、自分で問題解決へ調査から発表まで行う楽しさを知った。
この2つの経験から、\underline{興味のある分野を研ぎ究める}と同時に\underline{社会問題を解決して人々の生活を幸せ}にしたいと考えるようになった。

大学入学から研究活動を開始するまでの間、
熊本地震や米国大統領選挙でフェイクニュース問題が頻繁に取り沙汰された。
誤った情報が広まることで騙された人々によって多くの社会的損害が発生しており、
中には故意に作られたものも含まれていた。
あまりに速く拡散されるある種``広め得''な状況を、私はフェアではないと感じるようになった。
誤った情報が広まって風評被害が出る事例は古今東西起きているものの、
ことSNSが普及した現代社会では共有によって拡散のスピードが速く広くなる点に危機感を抱いた。
事実と異なる情報が訂正が入る前に広く拡散され、
\underline{騙されて誤った風評が残り不幸になる人を0にしたい}という考えがテーマ選定の大きな動機である。

修士研究を終えるまでの3年もの間、このテーマで研究を進めていくなかでフェイクニュースを自動で検出し利用者の拡散を抑制する難しさを認識した。
フェイクニュースは読者を騙すため精巧なものも多く、
真偽ラベルを付加するアノテーションを専門家以外に任せにくい上に、
判断結果をただ見せるだけでは読者を納得させる説得力を持たせにくい。
これらに対処するため、先行研究ではその解決に向けて自然言語処理に限らず幅広い分野の知見を取り入れている。
多くのアプローチが試されている中で、
大きな新規性をもたらすための\underline{新たな発想をいかに取り入れるかが重要}である点を修士研究までの3年間で痛感した。
その実現に向けて、他分野にも広く精通するためには\underline{国内外の研究者達と活発な議論を通して理解を深める}必要性も認識している。

また自動検出を利用者によるフェイクニュース拡散の抑止に繋げるためには、
利用者に納得できる形で提供する重要性も同時に認識している。
判断結果に説得力がないと、利用者による信頼を得られず拡散抑止への効果が薄れるためである。
研究そのものも同じく、社会に対してわかりやすい説明を行うことで更に提供モデルの効果も強くなる。
このように\underline{自己完結のみならず成果を他人に伝えるまでが研究}と考える。

最終的にはフェイクニュースの自動検出を発展させ、
ニュースやSNS上での拡散現象から利用者による拡散活動を解き明かし、
\underline{誤った情報で被害を受ける人を0にする}ことを目指す。
さらに今後はアカデミックポストとしてアウトリーチ活動も積極的に行うことで、
研究と同時に後輩学生・研究者・SNS利用者へ\underline{分かりやすい説明を通して持続可能な社会の発展へ成果を還元できる研究者}を目指す。

%end 目指す研究者像 ====================

\vspace{5mm}
\noindent
\graybf{(2)上記の「目指す研究者像」に向けて、特別研究員の採用期間中に行う研究活動の位置づけ}

%begin 研究活動の位置づけ ====================
特別研究員の採用期間中に行う研究活動のなか、
4-(2)で挙げた今後研究者としてさらなる発展のため必要と考えている要素の習得を通して、
\underline{学術研究で得た知見を直接SNS利用者を含む日本社会に還元}する研究者を目指す。

その実現に向け、査読付き国際会議ないしは国際論文誌への論文発表をはじめ、
国内・国際会議での口頭発表も積極的に行う。
また、自然言語処理コミュニティに限らず国内ニュースメディアと積極的に連携を行い、
フェイクニュースの自動検出に関連した共同研究の実現が理想である。

特別研究員として研鑽を重ねていき、
現状の研究への新たな発想を追加し、実現に向けて幅広い人々と議論を重ね、得られた成果を端的に説明することが、
\underline{能動的に一貫して社会課題を解決へ自ら導く研究者}として大成へと導く。
その実現の大きな足がかりが本研究計画である。
%end 研究活動の位置づけ ====================
\end{spacing}
\input{pieces/p05_my_ambitions_01}

