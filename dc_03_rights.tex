%#Split: 03_rights  
%#PieceName: p03_rights
% p03_rights_00.tex
\KLBeginSubjectWithHeaderCommands{03}{3}{人権の保護及び法令等の遵守への対応}{1}{F}{}{\DCPDFirstSubjectPageStyle}{\DCPDDefaultPageStyle}

\section{人権の保護及び法令等の遵守への対応}
%    <<最大 1ページ>>

% s09_rights
%begin 人権の保護及び法令等の遵守への対応 ====================
コメント取得を予定してしているSNSはTwitterである。
Twitter社は2020年3月より学術目的でTwitter APIの利用を自由化しているほか、
取得したツイートIDを含む情報をデータセットとして公開することも学術目的であれば認められている\cite{twitter_2020}。

また、先行研究が提供したデータセットを使用する場合は、提供者が示しているライセンスやポリシーを遵守する。

なお、学習済みモデルの公表は平成30年改正著作権法第30条4号により認められている。

ただし、本研究では主観評価実験としてSNSユーザを対象としたアンケート調査を予定している。
この調査により収集したデータは、個⼈の特定につながる情報を匿名化した上で解析を⾏い、
解析結果の公表に際しては、匿名化を⾏ったデータを⽤い、個⼈情報の漏洩防⽌に配慮する。

\vspace{1cm}
{\footnotesize
	\begin{thebibliography}{99}
		\setcounter{enumiv}{11}
		\bibitem{twitter_2020} Twitter開発者ポリシーを分かりやすくアップデート, 2020年3月11日. (最終閲覧日 2020年4月19日) \url{https://blog.twitter.com/developer/ja_jp/topics/tools/2020/DevPolicyUpdate.html}
	\end{thebibliography}
}
%end 人権の保護及び法令等の遵守への対応 ====================

% p03_rights_01.tex
\KLEndSubject{F}


